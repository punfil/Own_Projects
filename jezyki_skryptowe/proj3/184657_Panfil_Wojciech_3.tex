\documentclass[11pt]{article}
\usepackage[a4paper, margin=2.5cm]{geometry}
\usepackage[T1]{fontenc}
\usepackage{polski}
\usepackage{babel}
\usepackage{indentfirst}
\usepackage{listings}
\usepackage[colorlinks=true, urlcolor=blue, linkcolor=red]{hyperref}
\usepackage[document]{ragged2e}
\lstset{
	frame=l,
	basicstyle=\ttfamily,
	numbers=left,
	texcl=false,
	tabsize=1,
	breaklines=true,
	postbreak=\mbox{{$\hookrightarrow$}\space}
}
\sloppy

\title{
	\textbf{Języki skryptowe i ich zastosowania}\\
	Zadanie nr 3: Aplikacja z GUI - zegar}

\author{Wojciech Panfil, 184657}

\date{3 kwietnia 2024}

\begin{document}
	\maketitle
	
    \section{Opis programu}
	\justify
	Celem niniejszego projektu jest implementacja własnego zegara w formie
    aplikacji z graficznym interfejsem użytkownika w języku \texttt{Python}. Aplikacja powinna posiadać
    dwa interfejsy graficzne: jeden oparty o bibliotekę \texttt{PyQt}, drugi o bibliotekę \texttt{PyGTK}.
    W związku z lepszą dostępnością materiałów w internecie (poradniki, dokumentacja, itp.) zdecydowano użyć \texttt{PyQt} w wersji \texttt{5} 
    oraz \texttt{PyGTK} w wersji \texttt{3}.

    \section{Uruchomienie programu}
    W zależności od wartości parametru linii poleceń, program będzie uruchamiać wybrany przez użytkownika interfejs graficzny:
    \begin{lstlisting}[language=bash, caption={Uruchomienie programu z GUI opartym o bibliotekę \texttt{PyQT5}},captionpos=b]
    python3 ./clock.py -qt
    \end{lstlisting}
    \begin{lstlisting}[language=bash, caption={Uruchomienie programu z GUI opartym o bibliotekę \texttt{PyGTK3}},captionpos=b]
    python3 ./clock.py -gtk
    \end{lstlisting}
    W przypadku braku wyboru, uruchomione zostanie GUI napisane z użyciem biblioteki \texttt{PyQT5}.
    Program będzie wspierał jeden język interfejsu użytkownika - polski.

    \section{Funkcjonalność}
    Jako wzór do określenia funkcjonalności programu posłużył program "zegar" wbudowany
    w system Android (com.google.android.deskclock).

    Implementacja autora niniejszego sprawozdania posiadać będzie cztery zakładki:
    \begin{enumerate}
        \item Zegar (domyślna) - wyświetla aktualny czas z dokładnością do sekundy,
        umożliwia również wybór strefy czasowej z listy.
        Lista stref czasowych pochodzić będzie z biblioteki \texttt{pytz}.
        \item Stoper - umożliwiać będzie pomiar czasu z dokładnością do sekundy.
        Na głównym planie znajdować się będzie upłynięty czas, natomiast pod spodem
        trzy przyciski - Rozpocznij, Zatrzymaj, Wyczyść, służące odpowiednio do 
        rozpoczęcia pracy stopera, zatrzymania i wyczyszczenia zmierzonej wartości.
        \item Minutnik - umożliwiać będzie odmierzenie czasu z dokładnością do sekundy.
        Na głównym planie znajdować się będzie pole tekstowe z możliwością wprowadzenia czasu
        do odmierzenia w formacie HH:MM:SS.
        Jednocześnie pole to będzie wyświetlać upływający czas w przypadku włączenia minutnika.
        Dolny fragment okna przeznaczony zostanie na przyciski rozpoczęcia i zatrzymania odliczania.
        \item Budzik - umożliwiać będzie włączenie alarmu o określonej przez użytkownika godzinie.
        Na głównym planie znajdować się będzie lista włączonych alarmów. Na dole okna, poniżej listy znajdować
        się będą przyciski umożliwiające dodanie nowego lub usunięcie istniejącego alarmu. 
        Po wybraniu alarmu z listy i dwukrotnym kliknięciu na niego (lub dodaniu nowego) otworzy się nowe okno dialogowe, w którym
        użytkownik będzie mógł określić czas alarmu z dokładnością do sekundy, a także dni tygodnia, w których
        budzik ma być aktywny. Dodatkowo na dole znajdować się będzie przełącznik, który określać będzie,
        czy dany budzik jest włączony. W przypadku włączenia się alarmu,
        użytkownik zostanie powiadomiony o tym fakcie poprzez wyświetlenie powiadomienia (okna dialogowego), a budzik zostanie dezaktywowany.
    \end{enumerate}

    \section{Projekt interfejsu graficznego - \texttt{PyQT5}}
    Implementacja z użyciem biblioteki \texttt{PyQT5} oparta będzie o klasę \texttt{QWidget}.
    Na górze okna stworzony zostanie \texttt{QTabWidget}, w którym umieszczane będą kolejne zakładki.
    Przełączenie pomiędzy zakładkami następować będzie przez naciśnięcie wybranej zakładki. Aktualnie wybrana
    zakładka będzie wyróżniona.

    Pierwszą z nich będzie wspomniany wcześniej zegar. Na środku okna znajdować się będzie obiekt \texttt{QLabel},
    który wyświetlać będzie czas. Na dole okna umieszczona zostanie lista rozwijana z dostępnymi strefami czasowymi 
    \texttt{pytz\.common\_timezones} w formie \texttt{QComboBox}. Wszystkie obiekty tej zakładki zostaną zamknięte
    w \texttt{QVBoxLayout}.

    \texttt{QWidget} stoper wyświetlać będzie czas w formie \texttt{QLabel}. Przyciski sterujące
    pomiarem umieszczone poniżej będą instancjami klasy \texttt{QPushButton}. Przyciski "Rozpocznij", "Zatrzymaj" oraz "Wyczyść", celem
    zgrupowania ich na dole okna będą znajdować się w osobnym kontenerze \texttt{QVBoxLayout}. Wszystkie
    elementy będą dodane do wspólnego dla tej zakładki obiektu \texttt{QVBoxLayout}.

    Trzecia funkcjonalność - minutnik - będzie zrealizowana identycznie jak stoper, z wyjątkiem
    zastąpienia \texttt{QLabel} obiektem klasy \texttt{QTimeEdit}. Dzięki temu użytkownik będzie mógł
    w łatwy sposób wprowadzić żądaną wartość czasu do odmierzenia.

    Ostatni element programu - budzik - będzie składał się z centralnie umieszczonej listy alarmów
    \texttt{QListWidget}. Pod nią, na dole okna, znajdować się będą przyciski dodania ("Dodaj nowy alarm") i usunięcia ("Usuń alarm") zgrupowane przez
    \texttt{QVBoxLayout}. Całość zamknięta będzie w \texttt{QVBoxLayout}. Warto wspomnieć, że do wyświetlania komunikatu o włączeniu się alarmu wykorzystywany będzie
    \texttt{QMessageBox}. Zawierać on będzie przycisk \texttt{QPushButton} "Zamknij" oraz obiekt \texttt{QLabel} wyświetlający godzinę alarmu.
    Do edycji oraz dodania nowego budzika posłuży obiekt \texttt{QDialog}. W nim, znajdować się będzie pole \texttt{QTimeEdit} umożliwiające zmianę godziny budzika,
    pola \texttt{QCheckBox} służące aktywacji alarmu w określone dni tygodnia oraz włączenie alarmu, na samym dole będą natomiast przyciski \texttt{QPushButton} "Zapisz"
    oraz "Anuluj". Przyciski zgrupowane zostaną z użyciem \texttt{QHBoxLayout}, natomiast wszystkie obiekty z użyciem \texttt{QVBoxLayout}.

    Za odliczanie czasu, zarówno dla budzika jak i stopera oraz minutnika odpowiedzialny będzie \texttt{QTimer}.

    Na samej górze okna, powyżej paska zakładek, znajdować się będzie pasek \texttt{QMenuBar}. 
    Będzie on oferował dwie aktywności \texttt{QAction}: Przycisk "Plik" z przyciskiem "Wyjdź" znajdującym się na wysuniętej liście lub
    wyświetlenie krótkiego opisu aplikacji (przycisk "O aplikacji") w formie nowego okna z jednym przyciskiem "Ok".

    \section{Projekt interfejsu graficznego - \texttt{PyGTK3}}
    Implementacja z użyciem biblioteki \texttt{PyGTK3} oparta będzie o klasę \texttt{Gtk.Window}.
    Na górze okna stworzony zostanie \texttt{Gtk.Notebook}, w którym umieszczane będą kolejne zakładki.
    Przełączenie pomiędzy zakładkami następować będzie przez naciśnięcie wybranej zakładki. Aktualnie wybrana
    zakładka będzie wyróżniona.

    Pierwszą z nich będzie wspomniany wcześniej zegar. Na środku okna znajdować się będzie obiekt \texttt{Gtk.Label},
    który wyświetlać będzie czas. Na dole okna umieszczona zostanie lista rozwijana z dostępnymi strefami czasowymi 
    \texttt{pytz.common\_timezones} w formie \texttt{Gtk.ComboBoxText}. Wszystkie obiekty tej zakładki zostaną zamknięte
    w \texttt{Gtk.Box(orientation=Gtk.Orientation.VERTICAL)}

    \texttt{Gtk\.box} stoper będzie wyświetlać czas w formie \texttt{Gtk.Label}. Przyciski sterujące
    pomiarem umieszczone poniżej będą instancjami klasy \texttt{Gtk.Button}. Przyciski "Rozpocznij", "Zatrzymaj" oraz "Wyczyść",
    celem zgrupowania ich na dole okna będą znajdować się w osobnym kontenerze
    \texttt{Gtk.Box(orientation=Gtk.Orientation.VERTICAL)}. Wszystkie
    elementy będą dodane do wspólnego dla tej zakładki obiektu
    \texttt{Gtk.Box(orientation=Gtk.Orientation.VERTICAL)}.

    Trzecia funkcjonalność - minutnik - będzie zrealizowana identycznie jak stoper, z wyjątkiem
    zastąpienia \texttt{Gtk.Label} obiektem klasy \texttt{Gtk.Entry}. Dzięki temu użytkownik będzie mógł
    w łatwy sposób wprowadzić żądaną wartość czasu do odmierzenia.

    Ostatni element programu - budzik - będzie składał się z centralnie umieszczonej listy alarmów
    \texttt{Gtk.ListBox}. Pod nią, na dole okna, znajdować się będą przyciski dodania ("Dodaj nowy alarm") i usunięcia ("Usuń alarm") zgrupowane przez
    \texttt{Gtk.Box(orientation=Gtk.Orientation.VERTICAL)}. Całość zamknięta będzie w
    \texttt{Gtk.Box(orientation=Gtk.Orientation.VERTICAL)}. Warto wspomnieć, że do wyświetlania komunikatu o włączeniu się alarmu wykorzystywany będzie
    \texttt{Gtk.MessageDialog}. Zawierać on będzie przycisk \texttt{Gtk.Button} "Zamknij" oraz obiekt \texttt{Gtk.Label} wyświetlający godzinę alarmu.
    Do edycji oraz dodania nowego budzika posłuży obiekt \texttt{Gtk.Dialog}. W nim, znajdować się będzie pole \texttt{Gtk.Entry} umożliwiające zmianę godziny budzika,
    pola \texttt{Gtk.CheckButton} służące aktywacji alarmu w określone dni tygodnia oraz włączenie alarmu, na samym dole będą natomiast przyciski \texttt{Gtk.Button} "Zapisz"
    oraz "Anuluj". Wszystkie obiekty zostaną zgrupowane z użyciem \texttt{Gtk.Grid}.

    Za odliczanie czasu, zarówno dla budzika jak i stopera oraz minutnika odpowiedzialna będzie funkcja \texttt{GLib.timeout\_add\_seconds}.

    Na samej górze okna znajdować się będzie pasek \texttt{Gtk.Menu}. Będzie on oferował dwa elementy \texttt{Gtk.MenuItem}:
    Przycisk "Plik" z przyciskiem "Wyjdź" znajdującym się na wysuniętej liście lub wyświetlenie krótkiego opisu aplikacji (przycisk "O aplikacji")
    w formie nowego okna z jednym przyciskiem "Ok"..

\end{document}