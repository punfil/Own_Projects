\documentclass[11pt]{article}
\usepackage[a4paper, margin=2.5cm]{geometry}
\usepackage[T1]{fontenc}
\usepackage{polski}
\usepackage{babel}
\usepackage{indentfirst}
\usepackage{listings}
\usepackage[colorlinks=true, urlcolor=blue, linkcolor=red]{hyperref}
\usepackage[document]{ragged2e}
\lstset{
	frame=l,
	basicstyle=\ttfamily,
	numbers=left,
	texcl=false,
	tabsize=1,
	breaklines=true,
	postbreak=\mbox{{$\hookrightarrow$}\space}
}
\sloppy

\title{
	\textbf{Języki skryptowe i ich zastosowania}\\
	Zadanie nr 3: Aplikacja z GUI - zegar}

\author{Wojciech Panfil, 184657}

\date{3 kwietnia 2024}

\begin{document}
	\maketitle
	
    \section{O projekcie}
	\justify
	Celem niniejszego projektu jest implementacja własnego zegara w formie
    aplikacji graficznej w języku \texttt{Python}. Aplikacja powinna posiadać
    dwa interfejsy graficzne: jeden oparty o \texttt{PyQt5}, drugi o \texttt{PyGTK3}.

    \section{Uruchomienie programu}
    Program będzie uruchamiać wybrany przez użytkownika interfejs graficzny w zależności
    od wartości parametru linii poleceń:
    \begin{lstlisting}[language=bash, caption={uruchomienie GUI w PyQT5},captionpos=b]
    python3 ./clock.py -qt
    \end{lstlisting}
    \begin{lstlisting}[language=bash, caption={uruchomienie GUI w PyGTK3},captionpos=b]
    python3 ./clock.py -gtk
    \end{lstlisting}
    W przypadku braku wyboru, uruchomione zostanie GUI w PyQT5.
    Program będzie składać się z jednego okna i będzie wspierał jeden język interfejsu użytkownika - polski.

    \section{Funkcjonalność}
    Wzorem do określenia funkcjonalności programu posłużył program "zegar" wbudowany
    w system Android (com.google.android.deskclock).

    Implementacja autora niniejszego sprawozdania posiadać będzie cztery zakładki:
    \begin{enumerate}
        \item Zegar (domyślna) - wyświetla aktualny czas z dokładnością do sekundy,
        umożliwia również wybór strefy czasowej z listy wybory.
        Lista stref czasowych pochodzić będzie z biblioteki pytz.
        \item Stoper - umożliwiać będzie pomiar czasu z dokładnością do sekundy.
        Na głównym planie znajdować się będzie upłynięty czas, natomiast pod spodem
        trzy przyciski - Rozpocznij, Zatrzymaj, Wyczyść, służące odpowiednio do 
        rozpoczęcia pracy stopera, zatrzymania i wyczyszczenia zmierzonej wartości.
        \item Minutnik - umożliwiać będzie odmierzenie czasu z dokładnością do sekundy.
        Na głównym planie znajdować się będzie odmierzany czas, następnie na dole znajdować się
        będzie pole tekstowe z możliwością wprowadzenia czasu do odmierzenia w formacie HH:MM:SS.
        Dół przeznaczony zostanie na przycisk rozpoczęcia odliczania i zatrzymania.
        \item Budzik - umożliwiać będzie włączenie alarmu o określonej przez użytkownika godzinie.
        Na głównym planie znajdować się będzie lista włączonych alarmów. Na dole listy znajdować
        się będzie przycisk umożliwiający dodanie nowego lub usunięcie istniejącego alarmu. 
        Po wybraniu alarmu z listy i dwukrotnym kliknięciu na niego (lub dodaniu nowego)
        użytkownik będzie mógł określić czas alarmu z dokładnością do sekundy, a także dzień tygodnia, w których
        budzik ma być aktywny. Dodatkowo na dole znajdować się będzie przełącznik, który określać będzie,
        czy dany budzik jest włączony. W przypadku włączenia się alarmu,
        użytkownik zostanie powiadomiony przez wyświetlenie powiadomienia, a budzik zostanie dezaktywowany.
    \end{enumerate}

    \section{Implementacja - PyQT5}
    Implementacja z użyciem biblioteki PyQT5 opierać się będzie o klasę \texttt{QWidget}.
    Na górze okna stworzony zostanie \texttt{QTabWidget}, w którym umieszczane będą kolejne zakładki.

    Pierwszą z nich będzie wspomniany z zegar. Na środku okna znajdować się będzie obiekt \texttt{QLabel},
    który wyświetlać będzie czas. Poniżej umieszczona zostanie lista z dostępnymi strefami czasowymi 
    \texttt{pytz\.common\_timezones} w formie \texttt{QComboBox}. Całość obiektów zamknięta zostanie
    w \texttt{QVBoxLayout}

    \texttt{QWidget} stoper wyświetlać będzie czas w formie \texttt{QLabel}. Przyciski sterujące
    pomiarem umieszczone pod spodem będą instancjami klasy \texttt{QPushButton}. Przyciski, celem
    zgrupowania ich na dole będą znajdować się w osobnym kontenerze \texttt{QVBoxLayout}. Wszystkie
    elementy będą dodane do wspólnego dla tej zakładki obiektu \texttt{QVBoxLayout}.

    Trzecia funkcjonalność - minutnik - będzie zrealizowana identycznie jak stoper, z wyjątkiem
    zastąpienia \texttt{QLabel} obiektem klasy \texttt{QTimeEdit}. Dzięki temu użytkownik będzie mógł
    w łatwy sposób wprowadzić żądaną wartość czasu.

    Ostatni element programu - budzik - będzie składał się z centralnie umieszczonej listy alarmów
    \texttt{QListWidget}. Pod nią znajdować się będą przyciski dodania i usunięcia ułożone w
    \texttt{QVBoxLayout}. Całość zamknięta będzie w \texttt{QVBoxLayout}. Warto wspomnieć, że do edycji,
    dodawania nowego budzika, a także do wyświetlania komunikatu o włączeniu się alarmu wykorzystywany będzie
    \texttt{QMessageBox}.

    Za odliczanie czasu, zarówno dla budzika jak i stopera czy minutnika odpowiedzialny będzie \texttt{QTimer}.

    Na górze okna znajdować się będzie pasek \texttt{QMenuBar}. Będzie on oferował dwie aktywności \texttt{QAction}:
    wyjście lub wyświetlenie informacji o aplikacji.

    \section{Implementacja - PyGTK3}
    Implementacja z użyciem biblioteki PyGTK3 opierać się będzie o klasę \texttt{Gtk.Window}.
    Na górze okna stworzony zostanie \texttt{Gtk.Notebook}, w którym umieszczane będą kolejne zakładki.

    Pierwszą z nich będzie wspomniany z zegar. Na środku okna znajdować się będzie obiekt \texttt{Gtk.Label},
    który wyświetlać będzie czas. Poniżej umieszczona zostanie lista z dostępnymi strefami czasowymi 
    \texttt{pytz.common\_timezones} w formie \texttt{Gtk.ComboBoxText}. Całość obiektów zamknięta zostanie
    w \texttt{Gtk.Box(orientation=Gtk.Orientation.VERTICAL)}

    \texttt{Gtk\.box} stoper wyświetlać będzie czas w formie \texttt{Gtk.Label}. Przyciski sterujące
    pomiarem umieszczone pod spodem będą instancjami klasy \texttt{Gtk\.Button}. Przyciski, celem
    zgrupowania ich na dole będą znajdować się w osobnym kontenerze
    \texttt{Gtk.Box(orientation=Gtk.Orientation.VERTICAL)}. Wszystkie
    elementy będą dodane do wspólnego dla tej zakładki obiektu
    \texttt{Gtk.Box(orientation=Gtk.Orientation.VERTICAL)}.

    Trzecia funkcjonalność - minutnik - będzie zrealizowana identycznie jak stoper, z wyjątkiem
    zastąpienia \texttt{Gtk.Label} obiektem klasy \texttt{Gtk.Entry}. Dzięki temu użytkownik będzie mógł
    w łatwy sposób wprowadzić żądaną wartość czasu.

    Ostatni element programu - budzik - będzie składał się z centralnie umieszczonej listy alarmów
    \texttt{Gtk.ListBox}. Pod nią znajdować się będą przyciski dodania i usunięcia ułożone w
    \texttt{Gtk.Box(orientation=Gtk.Orientation.VERTICAL)}. Całość zamknięta będzie w
    \texttt{Gtk.Box(orientation=Gtk.Orientation.VERTICAL)}. Warto wspomnieć, że do edycji,
    dodawania nowego budzika, a także do wyświetlania komunikatu o włączeniu się alarmu wykorzystywany będzie
    \texttt{Gtk.MessageDialog}.

    Za odliczanie czasu, zarówno dla budzika jak i stopera czy minutnika odpowiedzialny będzie \texttt{GLib.timeout\_add\_seconds}.

    Na górze okna znajdować się będzie pasek \texttt{Gtk.Menu}. Będzie on oferował dwa elementy \texttt{Gtk.MenuItem}:
    wyjście lub wyświetlenie informacji o aplikacji.

\end{document}