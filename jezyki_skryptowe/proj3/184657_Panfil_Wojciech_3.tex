\documentclass[11pt]{article}
\usepackage[a4paper, margin=2.5cm]{geometry}
\usepackage[T1]{fontenc}
\usepackage{polski}
\usepackage{babel}
\usepackage{indentfirst}
\usepackage{listings}
\usepackage[colorlinks=true, urlcolor=blue, linkcolor=red]{hyperref}
\usepackage[document]{ragged2e}
\lstset{
	frame=l,
	basicstyle=\ttfamily,
	numbers=left,
	texcl=false,
	tabsize=1,
	breaklines=true,
	postbreak=\mbox{{$\hookrightarrow$}\space}
}
\sloppy

\title{
	\textbf{Języki skryptowe i ich zastosowania}\\
	Zadanie nr 3: Aplikacja z GUI - zegar}

\author{Wojciech Panfil, 184657}

\date{3 kwietnia 2024}

\begin{document}
	\maketitle
	
    \section{O projekcie}
	\justify
	Celem niniejszego projektu jest implementacja własnego zegara w formie
    aplikacji graficznej w języku \texttt{Python}. Aplikacja powinna posiadać
    dwa interfejsy graficzne: jeden oparty o \texttt{Qt}, drugi o \texttt{GTK}.

    \section{Uruchomienie programu}
    Program będzie uruchamiać wybrany przez użytkownika interfejs graficzny w zależności
    od wartości parametru linii poleceń:
    \begin{lstlisting}[language=bash, caption={uruchomienie GUI w PyQT5},captionpos=b]
    python3 ./clock.py -qt
    \end{lstlisting}
    \begin{lstlisting}[language=bash, caption={uruchomienie GUI w PyGTK3},captionpos=b]
    python3 ./clock.py -gtk
    \end{lstlisting}
    W przypadku braku wyboru, uruchomione zostanie GUI w PyQT5.

    \section{Funkcjonalność}
    Wzorem do określenia funkcjonalności programu posłużył program "Zegar" wbudowany
    w system Android.
    Program posiadać będzie cztery zakładki:
    \begin{enumerate}
        \item Zegar (domyślna) - wyświetla aktualny czas z dokładnością do sekundy,
        umożliwia również wybór strefy czasowej z listy wybory.
        Lista stref czasowych pochodzić będzie z biblioteki pytz.
        \item Stoper - umożliwiać będzie pomiar czasu z dokładnością do sekundy.
        Na głównym planie znajdować się będzie upłynięty czas, natomiast pod spodem
        trzy przyciski - Rozpocznij, Zatrzymaj, Wyczyść, służące odpowiednio do 
        rozpoczęcia pracy stopera, zatrzymania i wyczyszczenia zmierzonej wartości.
        \item Minutnik - umożliwiać będzie odmierzenie czasu z dokładnością do sekundy.
        Na głównym planie znajdować się będzie odmierzany czas, następnie na dole znajdować się
        będzie pole tekstowe z możliwością wprowadzenia czasu do odmierzenia w formacie HH:MM:SS.
        Dół przeznaczony zostanie na przycisk rozpoczęcia odliczania i zatrzymania.
        \item Budzik - umożliwiać będzie włączenie alarmu o określonej przez użytkownika godzinie.
        Na głównym planie znajdować się będzie lista włączonych alarmów. Na dole listy znajdować
        się będzie przycisk umożliwiający dodanie nowego alarmu. 
        Po wybraniu alarmu z listy (lub dodaniu nowego) użytkownik będzie mógł określić czas, w którym
        odtworzony zostanie dźwięk z dokładnością do sekundy. Dodatkowo, dostępny będzie
        przełącznik, który określać będzie, czy dany budzik jest włączony, a także lista dostępnych do
        odtworzenia dźwięków.
    \end{enumerate}

    \section{Implementacja - PyQT5}

    \section{Implementacja - PyGTK3}

\end{document}