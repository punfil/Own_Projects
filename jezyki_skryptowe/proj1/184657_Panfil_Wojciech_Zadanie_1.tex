\documentclass[11pt]{article}
\usepackage[a4paper, margin=2.5cm]{geometry}
\usepackage[T1]{fontenc}
\usepackage{polski}
\usepackage{babel}
\usepackage{indentfirst}
\usepackage{listings}
\usepackage[colorlinks=true, urlcolor=blue, linkcolor=red]{hyperref}
\usepackage[document]{ragged2e}
\lstset{
	frame=l,
	basicstyle=\ttfamily,
	numbers=left,
	texcl=false,
	tabsize=1,
	breaklines=true,
	postbreak=\mbox{{$\hookrightarrow$}\space}
}
\sloppy

\title{
	\textbf{Języki skryptowe i ich zastosowania}\\
	Zadanie nr 1: Test wydajności -- sprawozdanie}

\author{Wojciech Panfil, 184657}

\date{27 lutego 2024}

\begin{document}
	\maketitle
	
	\section{Testowana funkcja}
	\justify
	Celem niniejszego projektu była implementacja własnej funkcji
	konwertującej liczby do systemu ósemkowego w języku C oraz Python.
	W obu językach została ona nazwana \texttt{my\_oct}.
	Przyjmuje ona na wejście liczbę całkowitą, a następnie zwraca ją w
	postaci ciągu znaków z przedrostkiem \texttt{(-)0o}.

	\section{Implementacje funkcji}
	W języku C funkcja została nazwana \texttt{my\_oct} i zaimplementowana w poniższy sposób:
	\begin{lstlisting}[language=C, caption={Implementacja w języku C},captionpos=b]
	#define MAX_OCT_NUMBER_LENGTH 50
	unsigned char *my_oct(int64_t dec_num) {
		unsigned char *oct_num;
		bool is_positive;
		uint32_t length = 0;
		
		if (!dec_num) {
			return "0o0";
		}

		oct_num = calloc(1, MAX_OCT_NUMBER_LENGTH);
		if (!oct_num) {
			fprintf(stderr, "Not enough memory.\n");
			goto err;
		}

		is_positive = dec_num > 0 ? true : false;

		while (dec_num && length < MAX_OCT_NUMBER_LENGTH - 3) {
			oct_num[length++] = abs(dec_num % 8) + '0';
			dec_num /= 8;
		}

		oct_num[length] = '\0';
		strcat(oct_num, is_positive ? "o0" : "o0-");
		strrev(oct_num);

		return oct_num;
	err:
		return NULL;
	}
	\end{lstlisting}

	W języku Python została ona nazwana analogicznie i wyglądała następująco:

	\begin{lstlisting}[language=Python, caption={Implementacja w języku Python},captionpos=b]
	def my_oct(dec_num: int) -> str:
		'''
		Convert decimal number to octal representation.
		'''
		if not dec_num:
			return "0o0"

		prefix = "-0o" if dec_num < 0 else "0o"

		dec_num = abs(dec_num)
		oct_num_rev = ""
		while dec_num:
			oct_num_rev += str(dec_num % 8)
			dec_num //= 8

		return prefix + oct_num_rev[::-1]
	\end{lstlisting}
	
	Jak można zauważyć, jako odpowiednik typu \texttt{str} języka Python użyty został typ \texttt{unsigned char *}.

	\section{Testy poprawności}
	\justify
	Celem przetestowania poprawności zaimplementowanych funkcji, 
	wygenerowany został zbiór miliona unikalnych liczb mieszczących się w zakresie wartości \texttt{int64\_t} 
	z pomocą narzędzia \texttt{random.randint()} z biblioteki standardowej języka Python w formie pliku tekstowego
	(każda liczba w nowej linii). Uzyskane wyniki porównywane były do wartości zwracanych przez funkcję \texttt{oct}
	wbudowaną w bibliotekę standardową języka Python.
	\justify
	Dla języka Python zaimplementowane zostały unit test'y w framework'u \texttt{pytest}, które wczytywały wygenerowane
	liczby, a następnie porównywały wynik funkcji \texttt{my\_oct()} z \texttt{oct()}. W przypadku różnych wartości, test zgłaszał wyjątek.
	\justify
	Dla języka C stworzony został program, który jako argument przyjmował liczbę \texttt{int64\_t}, a rezultat \texttt{my\_oct()}
	umieszczał w standardowym wyjściu. Dzięki temu można było stworzyć podobny test jak dla funkcji \texttt{my\_oct()} zaimplementowanej w języku Python, który wykorzystywał
	bibliotekę \texttt{subprocess} do uruchamiania pliku wykonywalnego z odpowiednim argumentem, a następnie porównywał odczytany wynik z funkcją wbudowaną.
	
	\section{Testy wydajności}
		Celem uniknięcia przekłamań pochodzących z działającego w tle systemu synchronizacji czasu
		wykorzystany został zegar typu \texttt{monotonic}.

		\paragraph{Język C}
		Do pomiaru czasu w języku C, wykorzystane zostały funkcje zdefiniowane w pliku nagłówkowym \texttt{time.h}.
		Przed pomiarami wykonane zostało zapytanie o rozdzielczość pomiaru za pomocą:
		\begin{lstlisting}
		struct timespec resolution;
		clock_getres(CLOCK_MONOTONIC, &resolution);
		\end{lstlisting}
		Do pomiaru czasu wykorzystana została funkcja:
		\begin{lstlisting}
		struct timespec start;
		clock_gettime(CLOCK_MONOTONIC, &start);
		\end{lstlisting}

		\paragraph{Język Python}
		Do pomiaru czasu w języku Python, wykorzystała została biblioteka \lstinline{time}.
		Przed pomiarami wykonane zostało zapytanie o rozdzielczość pomiaru za pomocą:
		\begin{lstlisting}
		clock_info = time.get_clock_info('monotonic')
		\end{lstlisting}
		Do pomiaru czasu wykorzystana została funkcja:
		\begin{lstlisting}
		start = time.monotonic_ns()
		\end{lstlisting}

		Platforma testowa oparta była o procesor Intel Core i5 8600K o taktowaniu 3600MHz, 
		a wykorzystywanym systemem operacyjnym był Linux Fedora 38 z zainstalowanym interpreterem języka Python w wersji 3.11.6. 
		W przypadku obu języków rozdzielczość pomiaru wyniosła $1e-9$.
		Dokładność pomiaru nie została dokładnie ustalona z uwagi na trudność w dostępie do informacji.
		Jedynie dla języka Python 3.7 odnaleziono dokument PEP0564, według którego dokładność funkcji \texttt{time.monotonic\_ns()} ustalono na 84ns.
		Według wartości raportowanych przez system operacyjny:
		\begin{lstlisting}
cat /sys/devices/system/clocksource/clocksource0/current_clocksource
# Output: tsc
		\end{lstlisting}
		wykorzystywany był sprzętowy rejestr procesora, który był uaktualniany każdorazowo wraz z sygnałem CLK.
		W związku z powyższymi danymi, w sprawozdaniu przyjęto dokładność pomiaru na poziomie $1$s (błąd bezzwględny pomiaru w najgorszym przypadku).
		
		W eksperymencie, wykonano trzy pomiary czasu:
		\begin{itemize}
			\item pomiar $p_1$ – przed rozpoczęciem N-krotnego wykonania funkcji na wczytanych danych
			\item pomiar $p_2$ – po zakończeniu N-krotnego wykonania funkcji na wczytanych danych i przed rozpoczęciem n-krotnego wykonania pustej pętli
			\item pomiar $p_3$ – po zakończeniu N-krotnego wykonania pustej pętli (instrukcja \texttt{NOP})
		\end{itemize}
		Celem dobrania wartości $N$ - liczby powtórzeń, pod kątem nieprzekraczania $1\%$ przez wartość błędu względnego, wykonano pomiary czasu wykonania pojedynczej pętli ($p_2$ - $p_1$), 
		które wyniosły dla przykładowo wygenerowanych danych, po uprzednim odjęciu wartości ($p_3$ - $p_2$), odpowiednio:
		\begin{itemize}
			\item $3.676358073$s dla \texttt{my\_oct()} w języku Python,
			\item $0.072454523$s dla \texttt{oct()} w języku Python,
			\item $0.06633278$s dla \texttt{my\_oct()} w języku C.
		\end{itemize}

		Błąd bezwzględny pomiaru w najgorszym przypadku jest sumowany. W związku z tym, że trzeba wykonać dwukrotnie różnicę, co łącznie daje $4$ operandy, wynosi on $4$s.
		Celem osiągnięcia zamierzonej dokładności eksperyment powinien trwać zatem co najmniej 400 sekund. Dobrane wartości N oraz obliczony błąd względny przedstawione zostały w poniższej tabeli:
		
		\begin{table}[!h]
			\begin{tabular}{|c|c|c|c|}
				\hline
				\textbf{Funkcja} & \textbf{N} & \textbf{Czas działania} & \textbf{Błąd względny} \\
				\hline
				\texttt{oct()} w języku Python & $6500$ & $452.947763469$s & $0.883104$\% \\
				\hline
				\texttt{my\_oct()} w języku C & $6000$ & $466.296518647$s & $0.85782$\% \\
				\hline
				\texttt{my\_oct()} w języku Python & $115$ & $432.212319115$s & $0.92547$\% \\
				\hline
			\end{tabular}
		\end{table}
	
	\section{Analiza uzyskanych wyników}
	\justify
	Uzyskane dane dla $N=1$ wskazują, że najszybszą z testowanych funkcji jest ta ręcznie stworzona w języku C.
	Dla większej ilości iteracji przewagę zyskuje jednak implementacja zawarta w ramach biblioteki standardowej języka Python.
	Autorowi niniejszego sprawozdania wydaje się, że może to wynikać z kilku powodów:
	\begin{itemize}
		\item Technika zwalniania pamięci (GC uruchamiany co jakiś czas zamiast zwalniania każdorazowo blok po bloku,
		z drugiej strony przy zadaniach stricte obliczeniowych jego działanie może być kosztowne),
		\item Implementacja bazująca na operacjach bitowych (\href{https://github.com/python/cpython/tree/main}{według kodu źródłowego CPython'a}), zamiast dzielenia w pętli,
		\item Optymalizacje czasu kompilacji, np. wykorzystanie LTO,
		\item Brak wykorzystania operacji na stringach takich jak \texttt{strrev()}.
	\end{itemize}
	\justify
	W przypadku funkcji \texttt{my\_oct()} zaimplementowanej przez autora sprawozdania w języku Python, na gorszy wynik wpływ
	mają nie tylko czynniki wspomniane wcześniej takie jak nieoptymalna implementacja, ale także jego interpretowanie, zamiast kompilacji.
	Należy pamiętać, że funkcja \texttt{oct()} pochodząca z wbudowanej biblioteki jest skompilowana.


\end{document}