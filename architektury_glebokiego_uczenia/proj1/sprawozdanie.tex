\documentclass[11pt]{article}
\usepackage[a4paper, margin=2.5cm]{geometry}
\usepackage[T1]{fontenc}
\usepackage{polski}
\usepackage{babel}
\usepackage{indentfirst}
\usepackage{listings}
\usepackage[colorlinks=true, urlcolor=blue, linkcolor=red]{hyperref}
\usepackage[document]{ragged2e}
\lstset{
	frame=l,
	basicstyle=\ttfamily,
	numbers=left,
	texcl=false,
	tabsize=1,
	breaklines=true,
	postbreak=\mbox{{$\hookrightarrow$}\space}
}
\sloppy


\title{
	\textbf{Architektury głębokiego uczenia}\\
	Analiza użycia \texttt{FaceNet} do rozpoznawania ras psów}

    \author{
        Bartosz Leśniewski, 184783
        \and
        Jakub Kuczys, 184***
        \and
        Wojciech Panfil, 184657
      }

\date{5 czerwca 2024}

\begin{document}
	\maketitle

    \justify

    \section{Wprowadzenie i analiza problemu badawczego}
    Autorzy niniejszego sprawozdania postanowili podjąć się analizy systemu \texttt{FaceNet} opisanego w artykule
    \href{https://arxiv.org/abs/1503.03832}{\texttt{FaceNet}: A Unified  Embedding for Face Recognition and Clustering}
    w którym autorzy - Florian Schroff, Dmitry Kalenichenko oraz James Philbin podjęli się próby wykonania postępu
    w dziedzinie rozpoznawania twarzy. Jak wspominają sami autorzy, pomimo znaczących ostatnich postępów w tej dziedzienie
    skuteczne wdrażanie weryfikacji i rozpoznawania twarzy na dużą skalę stanowi poważne wyzwanie. Zaproponowany w artykule
    system FaceNet pozwala na unifikację podejścia do rozpoznawania, grupowania, weryfikacji. \texttt{FaceNet} uczy się mapowania
    obrazów twarzy do kompaktowej przestrzeni euklidesowej, w której odległości pomiędzy wektorami bezpośrednio odpowiadają
    miarom podobieństwa twarzy.
    Wewnętrznie, system ten wykorzystuje głęboką sieć konwolucyjną, wyszkoloną do bezpośredniej optymalizacji
    osadzenia w przestrzeni, co stanowi różnicę w porównaniu do uprzednich podejść do tego problemu. Do trenowania, wykorzystywana jest
    funkcja trójek składająca się z trzech twarzy, dwóch pasujących (kotwica i pozytyw) oraz jednej nie pasującej (negatywna).
    Celem treningu jest zatem minimalizowanie odległości pomiędzy wektorami pasujących do siebie twarzy, a maksymalizacja odległości
    do wektora odpowiadającemu niepasującej twarzy.

    Zaletą takiego podejścia jest znacznie większa wydajność pamięciowa - na twarz potrzeba jedynie $128$ bajtów (FaceNet uczy się przekładać
    twarz (embedding) na wektor w $128$ wymiarowej przestrzeni euklidesowej).

    Autorom tego systemu, na popularnym zbiorze danych "Labeled Faces in The Wild" udało się uzyskać wynik $99.64\%$ dokładności,
    co wyznaczało nowy rekord. Oznaczało to, że system obniżył poziom błędu w porównaniu do najlepszego wówczas opublikowanego wyniku o $30\%$.

    Celem niniejszego projektu była próba odtworzenia eksperymentu przeprowadzonego przez inżynierów Google'a, a następnie przetestowanie
    jego możliwości na znalezionym zestawie danych - \href{https://www.kaggle.com/datasets/wutheringwang/dog-face-recognition}{dog-face-recognition}
    udostępnionym w serwisie Kaggle. Dodatkowo, autorzy zdecydowali się spróbować dotrenować model na mordach psów, celem sprawdzenia skuteczności
    w rozpoznawaniu po kilkunastu epokach treningowych.

    Warto wspomnieć, że istnieje wiele implementacji FaceNet dostępnych w internecie. Najbardziej popularną jest \href{https://github.com/davidsandberg/facenet}{napisana przez David'a Sandberg'a}
    w Tensorflow, natomiast nie jest ona aktualizowana od ponad 6 lat. Do naszych eksperymentów zdecydowaliśmy się wykorzystać \href{https://github.com/timesler/facenet-pytorch}{gotową implementację FaceNet stworzoną przez Tim'a Esler'a},
    która opiera się na framework'u \texttt{PyTorch}, ponadto jest bieżąco utrzymywana.
    Zawiera ona nie tylko samą implementację, ale także dwa przetrenowane modele - jeden na zbiorze \href{https://paperswithcode.com/dataset/vggface2-1}{vggface2}, 
    a drugi na zbiorze \href{https://www.kaggle.com/datasets/debarghamitraroy/casia-webface}{CasiaWebface}. Dodatkowo, zawiera ona implementację MCTCNN, który pomocny jest w wykrywaniu
    twarzy, a następnie umożliwia docięcie zdjęcia do określonych wymiarów, tak aby była na nim wykryta twarz.

    \section{Szczegółowa architektura FaceNet}

    \section{Metodologia}


    \section{Wyniki eksperymentów, dyskusja i przyszła praca}

    \subsection{Reprodukcja eksperymentu zawartego w artykule - baseline}
    Pierwszy elementem eksperymentu było uruchomienie gotowego modelu \texttt{vggface2} na wspominanym wcześniej zbiorze \texttt{LFW}.
    W tym celu wykorzystany został \href{https://github.com/timesler/facenet-pytorch/blob/master/examples/lfw_evaluate.ipynb}{gotowy przykład dostępny w używanym repozytorium}.
    Do ewaluacji wymagany był dodatkowo plik z parami obrazów zwany dalej \texttt{pairs.txt}, który zawierał wygenerowane pary obrazów pasujących oraz niepasujących na podstawie zdjęć zawartych
    w zbiorze danych. Do jego wygenerowania została wykorzystana \href{https://github.com/VictorZhang2014/facenet/blob/master/mydata/generate_pairs.py}{gotowa implementacja generatora}.
    Poprawność przytoczonego narzędzia została sprawdzona z \href{vis-www.cs.umass.edu/lfw/pairs.txt}{plikiem udostępnianym ze zbiorem LFW}.

    Lokalne uruchomienie eksperymentu dało rezultat $99.28\%$ dokładności, co jest znacząco bliskie uzyskanej przez autorów artykułu wartości $99.64$\%.
    Dzięki temu autorzy potwierdzili, że ich środowisko jest działające, w związku z czym przeszli do kolejnej części eksperymentu.

    \subsection{Ewaluacja na zbiorze \texttt{dog-face-recognition}}

    \subsection{Trening na zbiorze \texttt{dog-face-recognition}}

    \subsection{Ponowna ewaluacja po treningu}

    \subsection{Przyszła praca}
    
    \section{Bibliografia}
    \begin{enumerate}
        \item \href{https://arxiv.org/abs/1503.03832}{FaceNet: A Unified Embedding for Face Recognition and Clustering}
        \item \href{https://en.wikipedia.org/wiki/FaceNet}{FaceNet basic information}
        \item \href{https://arxiv.org/abs/1604.02878}{Joint Face Detection and Alignment using Multi-task Cascaded Convolutional Networks}
        \item \href{https://medium.com/@culuma/face-recognition-with-facenet-and-mtcnn-11e77240adb6}{Face Recognition with FaceNet and MTCNN}
        \item \href{https://www.researchgate.net/profile/De-Rosal-Ignatius-Moses-Setiadi/publication/346417651_Study_Analysis_of_Human_Face_Recognition_using_Principal_Component_Analysis/links/605a0218458515e83467c633/Study-Analysis-of-Human-Face-Recognition-using-Principal-Component-Analysis.pdf}{Face Recognition using FaceNet Survey, Performance Test, and Comparison}
    \end{enumerate}
\end{document}